\chapter{简介}


近年来,深度学习技术在图像处理领域取得了巨大的成功。一方面得益于图像数据爆炸式的增多,另一方面,GPU技能飞跃式地提高使得快速的大规模计算成为可能。因此,在以前由计算瓶颈与数据量制约的深度神经网络,现在已经被广泛的使用。

在图像处理领域,与传统技术相比,深度学习技术往往有更好的效果,如图像识别,目标检测,语义分割等任务。实际上,这些都是颇具挑战性的难题,由于人们一直难以建立一个普适的数学模型,直到深度学习技术开始使用,才开始有了突破性的进展。甚至,在识别等任务中,神经网络的准确率已经超过人类\footnote{ImageNet上,Inception模型的错误率仅3.46\%,而一般情况下,人类则有5\%}。这是以前人们无法想象的,而在此其中起到关键作用的,是卷积神经网络(Convolutional Neural Netowrk, 以下简称CNN)。

CNN即是由多个卷积层,全连接层构成的神经网络\footnote{除此外还可能有池化层(pooling layer),以及为过拟合而加入的一些特殊层,如LRN(Local Response Normalization),因此CNN的定义并没有过于严格的要求},其中最为关键的卷积层,其作用可以概括为,提取图像的“特征”,而这些特征就可以用来作图像分类,描述等任务。与传统的神经网络(仅由全连接层组成的网络)相比,CNN具有更少的参数,这使得其训练所需的时间大大减少;其次,由于CNN卷积作用在图像上各个小区域,其提取出的特征具有更好的局部性,并且使得网络更加容易训练。

本论文的目的是做药盒的识别,即输入一张药物包装盒的图片,输出该药物的信息。考虑到问题的复杂性,现阶段本文将此问题简化为简单的分类问题,即是,对于给定的药物包装盒的图片,输出其生产厂商名。而将生产厂商名限制在预先给定的几个中后,这就变成一个经典的CNN图像分类问题 。

在过去十年中,人们提出了大量的CNN结构来解决图像分类问题,如Alexnet, VGG-net,以及当今识别率最高的Inception。一个显著的增长首先表现在网络的层数(如图),而随之带来的即是高强度的计算能力要求和大量的训练时间。比如Inception V3在一台个人笔记本上(配置GTX 1060显卡),可能需要好几天,甚至几个星期来完成训练。因此,在实验中,本文选择了较小规模的Alexnet,而这也需要将近一天的时间来训练(使用Nvidia Titan X Pascal显卡)。