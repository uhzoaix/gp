\chapter{卷积神经网络}

\section{神经网络简介}

在介绍神经网络之前,我们先重新规范化下问题。给定N组训练数据,$\{x_i, y_i\}_{i=1}^N$,其中$x_i \in \mathbb{R}^4, y_i \in \mathbb{R}$,即$x_i$是大小为$(n, w, h, c)$\footnote{n代表图像数量,w,h为图像的宽高,一般要求w,h相等,c则是图像的通道数,如RGB图像有3个通道,而灰度图像仅有一个通道}的四维张量,一般即代表输入的所有图像,而$y_i$则是图像的正确标签。现在需要解决的即是一个函数拟合的问题,即找到$F$,在给定的损失函数下,使如下总误差最小:

\[
L =\sum_{i=1}^N loss(F(x_i), y_i)
\]

在一般的函数拟合问题中,$F$所属的函数空间会被限制,如限制到给定最高次数的多项式空间。而在深度学习领域中,我们则是将$F$限制在所谓“神经网络函数”空间\footnote{当然并没有如此的定义,神经网络的内涵如今还在扩充中,并不能严格的定义它},其特征就是$F$是几种预先给定形式的函数的复合。比如,全连接层是具有如下形式的函数:

\[
	y = \sigma (x \cdot W + b)
\]

其中,x,y为输入输出的二维张量,形状分别为$(num,\  n)$与$(num,\  m)$,$W \in \mathbb{R}^{n \times m}, b \in \mathbb{R}^m $均是待定参数,$\sigma$是激活函数\footnote{神经网络中的非线性函数,常见的有Sigmoid函数,$S(t) = \frac{1}{1 + \exp(-t)}$ 与ReLU函数,$f(t) = \max(0, t)$},在这里作用张量,即分别作用到各个分量上。

可以看出对全连接层而言,$W,b$完全确定了函数本身。因此,在优化损失函数$L$时,所需优化的变量即是所有网络层的参数,这便简化了问题,而且由于激活函数的导数易于计算,使得在计算机上实现快速求解成为可能。

而损失函数的选择一般是依赖于问题的,比如,对于图像分类问题而言,人们习惯选择交叉熵(Cross Entropy)来作为损失函数,即总误差L如下定义:
\[
	Why people choose cross entropy
\]

在得到总误差的完整表达式后,我们就能将它作为目标函数,以数值优化的方法求得其最小值,更新其参数,这种优化求解,更新参数的过程在深度学习中即被称为训练。我们将在之后的章节更加详细的讨论关于优化的话题。

\section{CNN基本概念}

卷积神经网络,与一般的神经网络相比,多了卷积层的结构。下面将详细的介绍卷积神经网络中常见的概念,包括卷积层,池化层及为防止过拟合而进行的操作等。然后再说明对于目标函数的数值优化方法,并介绍其实现的数值原理,即误差向后传播算法(Backward Propagation of Errors)。

\subsection{建立模型}

一个简单的CNN数学模型的建立,即是各个层函数复合而成,这可以形象的描述为“将各个层按顺序叠起来”,为此我们要求每一层输入张量与其上一层输出张量的形状相同。如图(图片)即是LeNet的网络结构,它由2层卷积层(后接池化层)与2层全连层组成。在前面神经网络简介的章节里,我们已经介绍了全连层的概念,下面将介绍其他在CNN中常见的函数(层)。

\textbf{卷积层} \ 卷积层的概念与卷积运算密切相关。对于两个$\mathbb{L}^2$中的函数$f,g$而言,其卷积运算产生出新的函数按如下定义:
\[
	f * g \ (t) = \int_{\mathbb{R}} f(x) g(t-x) \  dx
\]

若f,g是二维张量,我们同样可以定义离散卷积:
\[
	f * g \ (x, y) = \sum_{i, j \in N(x,y), bdd(f)} f(i,j) \ g(x-i, y-j)
\]

这里$i,j \in N(x,y)$表示所有与$(x,y)$相邻且在f范围内\footnote{严格来讲为,$\{ \forall (i,j),|x-i| \le 1, |y-j| \le 1 \}$}的$(i,j)$。在图像处理中,高斯模糊即应用了离散卷积。一个灰度图像可以自然的看成一个矩阵f,然后定义一个方阵g\footnote{常见取的大小为3x3,5x5等,取决于具体应用,一般长宽为奇数},称作卷积核,其值由二维高斯函数确定\footnote{去查,与上个footnote合并},由于有限支集高斯函数具有低通滤波的特性,因此f和g做卷积后就能去除f的高频信息,如细节,噪声等。

与高斯模糊类似,卷积层函数也同样定义了自己的卷积核,并进行卷积操作得到新的张量。然而与前者相比,主要有如下几点不同:
\begin{enumerate}
	\item 卷积层可以有多个卷积核,每个卷积核对应输出张量的一个通道
	\item 该卷积核的值是未定参数,需要由误差函数优化得到
	\item 输入张量有多通道时,将各个通道做卷积操作后的值加起来作为结果
	\item 进行卷积操作后,需加入待定偏差参数b,然后被激活函数作用
\end{enumerate}

为了阐明清楚这些不同,我们写出其表达式。假设输入的一张图像$I$,大小为$(w_I, \ h_I, \ c_I)$,卷积核为$W^C = {(w^C_{ij})}_{k \times k}$,k为给定的卷积核大小,输出张量J,大小为$(w_J, \ h_J, \ c_J)$,则有如下表达式:
\[
	J(x,y,C) = \sigma(\sum_{1 \le c_1 \le C_I} I(\cdot, \cdot, c_1) * W^C \ (x,y) + b^C)
\]

其中,$C$的数量由外部给定,$W^C,b^C$为待定的参数,$\sigma$为激活函数。由此可以看出,卷积层的参数远少于全连层,这使得训练的时间大大减少,并且,卷积操作只涉及一小片区域,因此卷积层作用的结果更易体现局部特征(图,可视化的结果)。关于卷积层,还可以调整步长(strides)与填充(padding)等参数,我们放到附录中说明。

\textbf{池化层} \ 池化层(Pooling Layer),常见的有极大池化层(Max Pooling)与平均池化层(Average Pooling)

\textbf{防止过拟合的操作} \ 过拟合现象即在训练集上的误差很小,却在测试集上的误差变得较大。举例来说,现在要训练一个识别动物的CNN模型,但由于训练时过拟合,导致模型仅在训练集的图片上误差较小,而在新的同种类动物的图片上测试时,却无法识别。在Alexnet中,作者Krizhevsky应用了两种防止过拟合的操作,LRN与Dropout。

LRN(Local Response Normalization)

Dropout

\subsection{优化与误差向后传播算法}