\chapter{相关工作}

上世纪90年代,LeCun在MNIST\footnote{Mixed National Institute of Standards and Technology,一个业界流行的手写数字数据库}上使用了LeNet,使得在其上的准确率提高到99\%。然而,由于受到当时计算能力的制约,CNN的技术并没有被进一步发展。直至2012年,Krizhevsky提出了Alexnet,并在当时计算机视觉领域的一项竞赛\footnote{ILSVRC2012,测试数据集为ImageNet}中取得了第一名,其测试Top-5误差\footnote{即给一张图片,如果输出的5个预测能猜对它的正确标签,那么就判定为正确}达到了15.4\%,而第二名则有26.2\%。从网络结构上看,Alexnet拥有5个卷积层,这使得它对图像特征的表达能力相对于LeNet大大增强,并且能处理更高分辨率的图像。此外,为了解决模型过拟合的问题,Alexnet对全连接层加入Dropout操作,这在以后被证明处理过拟合时十分有效。

随后,CNN进入了高速发展的时期。2013年,Zeller与Fergus提出了ZFnet,其本身是Alexnet的改进,其错误率降低到11.2\%,并且所需训练的样本由Alexnet的1500万降到130万,而它的网络结构并未太大变化。ZFnet的另一个重要贡献在于,提出用于可视化网络中的参数与特征的技术DeconvNet。2014年,Simonyan与Zisserman提出了VGG-net,并达到了7.3\%的误差。与之前的网络结构相比,VGG-net并没有引入新的结构与函数,而是大量增多卷积层(共19层卷积层),使网络变“深”,其代价则是需要2到3个星期时间来训练网络。

而到了2015年,Google提出的GoogLeNet与微软提出的ResNet,大幅刷新了之前的记录,GoogLeNet的错误率达到6.7\%(在改进的版本Inception v3中,达到3.46\%),同时ResNet达到了3.6\%,而一般人类的错误率仅在5\% \textasciitilde 10\%之间。GoogLeNet不同于以往的CNN,由卷积层,全连层等直接顺序叠加,对于前一层的输入,它并行的进行多种卷积与池化操作,并最后聚合起来,这使得网络变“宽”。另一方面,GoogLeNet舍弃了全连层,并以平均池化层(Mean Pooling Layer)代替,这使得需训练的参数大大减少。而ResNet则换了一种思路,它借鉴了残差学习中的思想,提出了残差块的结构,用于拟合输入输出之间的差。ResNet即是残差块的大量叠加,并最终达到152层。然而作者也指出,单纯的增加层数只会降低准确率,并更可能引起过拟合,他们尝试了1202层的网络,而其准确率确实不升反降。